\section{Syntax}

This section describes the syntax of \tessla.

\subsection{Basic Syntax}

As basic syntactic elements of \tessla we consider a set of \emph{types} $T$, \emph{names} $\mathcal{N}$ and \emph{function symbols} $\mathcal{F}$.
Formally, the latter are two families $\mathcal{N}=(N_t)_{t \in T}$ and $\mathcal{F}=(F_{(t_1…t_m)t_{m+1}})_{t_1…t_{m+1} \in T^+}$ of distinct sets of symbols that correspond to a specific type or signature, respectively.
We refer to the tuple $Σ=(T,\mathcal{N}, \mathcal{F})$ as a \emph{signature} providing the basic syntactical elements of our specification language.

Built from those we define \emph{terms} $x$ of type $t \in T$ and sequences of terms $seq$ by the grammar
\[\begin{array}{rl}
x &::= n \mid f_c \mid f(seq) \mid x : t\\
seq &::= x \mid x,seq
\end{array}\]
where $n \in N_t$, $f_c \in F_{()t}$ (i.e., constants) and $f \in F_{(t_1…t_m)t}$.

\subsection{Syntactical Extensions}

We consider three syntactical extensions to the base syntax presented above. The first is an \emph{on} operator, the second are \emph{infix operators} and the third are \emph{named arguments}.

In the \emph{on} operator only stateless functions are allowed. Therefore, for a set of function symbols $F_{(t_1…t_m)t}$ let $F^s_{(t_1…t_m)t} \subseteq F_{(t_1…t_m)t}$ be the set of function symbols of stateless functions. Hence the \emph{terms} $x$ of a type $t \in T$ including the \emph{on} operator can be defined by the grammar
\[\begin{array}{rl}
x &::= n \mid f_c \mid f(seq) \mid \operatorname{on} f_s(seq_s) \mid x : t\\
seq &::= x \mid x,seq \\
y &::= n \mid f_c \mid f_s(seq_s) \mid \operatorname{on} f_s(seq_s) \mid y : t \\
seq_s &::= y \mid y,seq_s
\end{array}\]
where $f_s \in F^s_{(t_1…t_m)t}$ and the rest is defined as before.

