\section{Introduction}

Outline

\begin{enumerate}
  \item purpose, motivation
  \begin{itemize}
    \item online processing of data
    \item monitoring of trace properties, specifically execution traces of programs
    \item Functional reactive programming as related concept
  \end{itemize}
  \item What you describe with a \tessla specification
  \begin{itemize}    
    \item input, output streams
    \item application of functions, composition of function
    \item 
  \end{itemize}
  \item Modelling data in terms of streams
  \begin{itemize}
    \item timing model ($ℝ$,$ℕ$,$ℚ$,…), restrictions to streams with discrete set of time stamps (event streams) or peace-wise constant streams (continuous stream)
    \item continuous streams and event streams    
   \end{itemize}
   \item Functions on streams and desired properties (in general)
   \begin{itemize}
     \item small examples
     \item causality, statefulness, time invariance 
     \item (composition lemmata)
   \end{itemize}
   \item \tessla syntax
   \begin{itemize}
      \item base grammar with functions and type annotations
      \item syntactical extensions: infix operators, named arguments, the “on”
   \end{itemize}
   \item Types
   \begin{itemize}
     \item Generic types
     \item Coercion 
   \end{itemize}
   \item Functional semantics of operators, small examples
   \item Larger example/case study
   \begin{itemize}
     \item producer/consumer, ring buffer, …
   \end{itemize}
   
   
\end{enumerate}
